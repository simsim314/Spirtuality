\documentclass{article}
\usepackage{hyperref}
\usepackage{blindtext}
\usepackage{ulem}
\usepackage[T1]{fontenc}
\usepackage[utf8]{inputenc}
\usepackage{amsmath}
\usepackage{graphicx}
\usepackage{hyperref}
\usepackage{amsfonts}
\usepackage{amsmath}
\usepackage{amssymb}
\usepackage{ mathrsfs }

\title{\Huge Mathematical Theology}
\author{Michael Simkin, Alon Titelman}
\begin{document}
	\maketitle
	\section{Introduction}
	Theology and the notions of god, oracle, prophet, spirit, superhero etc. has fascinated people for thousand of years. We're suggesting a mathematically consistent framework of theological claims that can be decided based on different set of axioms inside the framework.\\
	
	In order to provide the theological framework inside physical reality we need the basic axiomatic claim that everything inside the physical reality is computable. It's still debatable weather our universe can be analyzed as finite or infinite tape Turing machine. Although we start from computational axioms as the reader will see, computability inside virtual reality can be quite complex and allow to formulate "supernatural" beings inside it to have mathematical meaning. As finite states Turing machine theory is well defined, we start from providing an infinite Turing machine framework which we will define in mathematical therms \footnote{ See also Joel David Hamkins and Andy Lewis \href{http://www.sharelatex.com}{Infinite turing machines} (1998) } . Then we will provide several proofs about the infinite Turing machines (ITM). We will postulate supernatural beings as creatures inside virtual reality with some level of access to the printed results of ITM - we call the oracle. 
	\section{Formal definitions}
	
	We postulate converging Turing machine as regular turing machine that converges in infinity. We also postulate a printer to which the Turing machine can write an infinite precision result. \\

	\textit{Definition: Converging Turing machine}, is a regular Turing machine where $\forall$ cell x $\in$ T (the tape) , $\exists$ n $\in$ $\mathbb{N}$ such that: $\forall$ operation index p > n, x is not changed. $\square$ \\
	
	
	\textbf{Theorem 1.1:} Every Turing machine is equivalent to converging Turing machine. \\ \\
	\textbf{Proof:}
	We will start from the proof from noticing that every Turing machine is equivalent to one sided Turing machine as negative numbers can be represented as odd indexes of the tape and positive as even. To show that every positives tape TM is equivalent to converging TM, we will stop the machine after N operations (for large N) and copy the tape and the head +N to the right. As our TM is working only on positives, it doesn't know where the 0 is, thus it will not move left from index N. As we're doing it every N operations, after i steps the tape will remain static for all x < iN. Thus the cell x will remain static after {\Large$\frac{x}{N}$} such copy operations. $\blacksquare$\\
	
	Thus in infinity the converging Turing machine will generate a constant tape and the tape is well define in $\omega$. Other than the tape we will use an infinite printer. \\
	
\textit{Definition: Infinite printer} is a function 	
	\begin{align*}
f:[0,1] \to \{0,1\} \  where \ \  
\|\{x\ | \ f(x) = 1\}\| = \aleph_0
\end{align*}
For which given an interval [a, b] a query Q: 
	\begin{align*}
	Q(a, b) = \{one \ element \ x\ | \ x \in [a, b] \ and \ f(x) = 1\}. \ \ 
	\end{align*}\\ While Q can be computed in O(1) time $\square$\\
	
Now that we know what is a Turing machine that works $\omega$ time and we can print a result into some place, we're ready to define ITM: \\
	
\textit{Definition: Infinite Turing machine (ITM)} is converging Turing machine which is also capable to send messages to infinite printer. $\square$ \\

Notice everything in this model is converging. The tape, the infinite list of printed numbers. The reading head might not converge, but we are not using the state of the head at any stage only the printed result which has converged. \\

\textit{Definition: Supertask} is a function call from ITM A to generate ITM B which will execute the code up to $\omega$ but in finite time for ITM A $\square$ \\

Supertask is basically equivalent to throwing a TM into black whole, it will make computations and send signal through photons to us (for us a finite time will pass for the TM infinite). If the infinite printer is not physically possible to implement, notice that most of the time we will not use this property of the printer (we usually can manage with ITM and finite printer). Yet it's important to notice the infinite printer is also well defined. \\

Now let us address to the ITM as an oracle. \\

 Oracle is an entity that has a result of any simulation that can occur in finite space of possibilities, and it will have the knowledge of the result of any possible outcome from the interaction with it. \\ 

\textit{Definition: Oracle} is function Q, given initial finite state $s_{t_{0}}$, and deterministic evolution rule $r:s_{t_{0}} \to s_{t_{0}+\epsilon}$
\begin{align*}
Q(s_{t_{0}}, r, t) \to s_{t} \ \forall \ t > t_{0} \ ,while \ Q \ is \ executing \ in \ O(1) 
\end{align*}

\textbf{Theorem 1.2:} ITM with supertasks and infinite printer is capable to generate an oracle. \\\\
\textbf{Proof:} We will start our calculation from generating a supertask that will print a list of all possible tapes. As all tapes are located in a tree, this is basic tree traversal algorithm to depth N done iteratively. Now we will use another supertask, to determine the outcome of every finite tape. This will be done by simulating N Turing machines for one step, and we will open another virtual (N+1)th TM. Thus eventually we will run all tapes simultaneously, for infinite amount of time and get the printed result from every tape run in $\omega$ time. Thus we've generated the simulation of all possible finite input tapes in $\omega$ time. Now to get an oracle we only need to query from the infinite printer the simulation with constant time. In order to do that because everything is finite we can query from inside an interval [a0,a1], where a is the simulation of me asking the oracle something getting the result acting upon it, and we can do it to any depth we want because everything is finite so this is basically a finite binary very long number that we query from the infinite printer at [a0, a1] - in this interval there will be automaton starting to evolve state st0 at point t, and the result can be only one number because we forced the tape to generate for us single result at this point. And we can check the result of this simulation from the oracle. 
 $\blacksquare$ \\

What is important to notice that any conversion of rules that we define can always be simulated inside some TM where the rules defined in some input tape as we must use universal TM. Thus for any situation there would be a very small interval from inside [0,1] which will define the rules of the simulation as a long number 0.010101011101 etc. 

	\section{Simple Oracles}

While the oracle knows everything there is to know about finite deterministic reality, the oracle response to person with query can be limited in certain ways. For example we can generate an oracle that can answer only yes/no to any given query. We will now formulate an equivalent to halt problem for the oracle. \\

\textbf{Theorem 2.1:} It's impossible to construct only truth speaking yes/no oracle. \\\\
\textbf{Proof:} Lets come to the oracle and ask: would I raise my hand in the next minute? If the oracle answers yes I will not raise my hand, and if the oracle says no I will raise my hand $\blacksquare$ \\ \\

The reason this query haven't been solved correctly is because the $simulation^{n}$ sequence (simulation inside a simulation etc.) will not converge. But if there is a convergence the oracle is capable to find the convergence inside the space of two answers yes/no and report it, using only 2 levels of simulation depth. \\

\textbf{Theorem 2.2:} It's possible to construct only truth speaking yes/no/else oracle, that can answer any solvable convergent question. \\\\ 
\textbf{Proof:}
The oracle will find a query where he answered yes, and a query where he answered no. If after a while the answer remains the same, the oracle will report it. If the answer is alternating the oracle will report "else".$\blacksquare$\\ 

Notice the query space is concerning any physical and personal issues. The oracle can't answer any mathematical question, he can only simulate anything humans can know about math. \\

\textbf{Theorem 2.3:} The query itself can influence the future i.e. it's possible that the answer would converge to yes if you didn't came to the oracle but after you came to the oracle the only convergent answer would be no. \\ \\
\textbf{Proof:} This is trivial. $\blacksquare$\\


\textbf{Theorem 2.4:} It's possible to get both answers truth. 
\\ \\
\textbf{Proof:} For example I can trust the oracle and do what he tells me to do - this will converge in both cases. $\blacksquare$\\

\section{Subject and Reality}

Before we move further to advanced oracles, we need to define mathematically some therms like subject, reality and perception of reality. \\

As reality is just a TM tape state which changes with time, the subject is obviously part of the TM tape state. But the reality starts to be interesting if there is a subjects in it. Otherwise it's just a state as any other state. \\ 

\textit{Definition: Subject} $A$ is a tuple $(t, T_{A}, S_{A})$. Where t $\in [0,1]$ representing some time t, $T_{A}$ is a TM tape representing some objective state and $S_{A} \in [0,1]$ which we call intrinsic state of the subject. $\square$ \\

\textit{Definition: Language}. Given two subjects A, B, the language is a subset of numbers $\mathscr{L} \subset \mathbb{N}$, and set of functions $Q_{A}^{i}: S_{A} \to \mathscr{L}$, and $P_{B}^{j}:\mathscr{L} \to S_{B}.$ The numbers from $\mathscr{L}$ called \textit{messages} which pass through A and B $\square$\\

\textit{Definition: The identity axiom}. Given subject $A$ and two states of tape x, y. The identity function $I_A(x, y)\to\{0,1\}$ representing weather the subject $A$ identify x and y to be the same. $\square$\\

\textit{Definition: The category axiom}. Given subject $A$, state x and category list $\mathbb{C}$. The category function $F^A(x)\to C_{x}^{A}\ where \ C_{x}^{A} \subset \mathbb{C}$ is called category axiom. Representing all categories x belongs to, for subject $A$ \ \ $\square$\\

\textit{Definition: Communication}. Given $(A, B, L, f_{A}, f_{B})$ where  $A$, $B$ two subjects, $L$ is language, and $f_A:S_A \to [0, 1] \ and\ f_B:S_B \to [0,1]$ the pass of messages between $A$ and $B$ using $L$, is increasing $f_A(S_A)\ and\ f_B(S_B).\ \  \square$\\

The subjectivity axioms claim that our words about our internal state are in some correlation with what we actually feel. As if there is no way to express the subject part at all, then there is no point to talk about him. But because we do have a language of our feelings, we can say the subject is a special part of the code which somehow becomes "aware". We should not ignore it nor accept it as a magic, from the other hand we should understand it's a special case - it's not another 0 or 1 as changing the function $H$ is the only thing with intrinsic value in a physical reality. \\

From the identity and category axioms follows that subjects have internal representation of reality i.e. what any specific subject recognizes as "being the same" or "being similar".\\



\section{Advanced Oracles}

Until now we've been constraining our oracle to answer only very limited set of answers, while checking only if they could speak the truth. What if we let the oracle to answer strings, and hold any set of values? This brings us to voracle. \\ 

\textit{Definition: Readable String} is a string returned by the oracle which also confirmed by the simulation the user will read. \\

\textit{Definition: $\mathcal{O}$} is a set of all readable strings for a specific query Q. \\

\textit{Definition: Voracle} is an oracle with preference function V. such that 
\begin{align*}V: s \to [0,1] \ for \ s \in \mathcal{O}. \ While \\
\ answer = max\{V(s) \ \forall s \in \mathcal{O}\}.\ \ \ \ \  \square
\end{align*} 

In this section we will explore different options for V, and briefly discuss the limitation of $\mathcal{O}$ \\

As we have only two answers for the simulation inside a simulation etc. s the oracle gets it easy, as it needs to check only finite amount of cases. But the correct answer might illuse him. \\

\textbf{Theorem 4.1:} Voracle might not know the answer himself
\\ \\
\textbf{Proof:}
Imagine there exist some irrational number which provides a non stable solution for the equation at hand i.e. what will maximize my happiness if infinities were allowed in my reality. i.e. if you copy my finite state into [0,1] and allow small not noticeable by me changes applied to my state, which would be equivalent to meeting more powerful oracle. Then current oracle will not be able to answer what is the solution to the equation of maximizing some function V, because he had only simulated convergent series of me cumming to him, but because the solution is not stable no $\aleph_0$ simulations will be enough to solve the partial differential equation. i.e. there might be non computable solutions to the problem  $\blacksquare$\\ (*need to ask physicists/mathematicians) \\

Now we will define several oracles that will become the foundation for the next chapters of this essay. The closest to the truth speaking voracle i.e. Toracle. The happiness maximizing voracle i.e. Horace. And finally the voracle which returns a compilable string that can answers practical coding queries if TM and robotics are available or you're inside a simulation and the source code is available to you i.e. Soracle. \\

The toracle is a voracle that tries to minimize the difference between the perceived answer by the subject and the actual state of reality. To understand this better and to define the therm properly we will need to postulate more definitions about the subject. \\


\textit{Definition: Toracle} is a voracle that returns to the subject $A$ string  $s \in \mathcal{O}$ such that given random sentence about reality the probability to make wrong judgment decrease optimally. $\square$ \\

Lets assume the subjects sees N bits that represent reality. i.e. lets take all sentences about reality distributed evenly in the space of words, which might be judged uniquely by the oracle as correct or incorrect. For example the sentence dragons are flying in china now" is clearly wrong and the oracle knows it's not true. So although all the therms like dragon and flying are all subjective (requiring the identity axiom), the oracle can understand the definition of those therm by the subject and identify false sentences that were recognize as true, true sentences that were missed or recognized as false. Toracle provides a response to give the subject the correct amount of sentences that he will read and recognize his mistakes about reality in the fastest way possible such that after the subject will learn the Toracle response the amount of sentences that he would be mistaken about will be minimal. 

Special case Toracles might include queries with limited amount of words in them (or amount of time will needed to learn the material), or to the contrary the Toracle that writes all the correct answers about reality for average human in a paste that will take us million years to learn. 

The Horacle is now very easy to define. 

\textit{Definition: Horacle} is a voracle that returns to the subject $A$ string  $s \in \mathcal{O}$ such that the integral of the feedback about the "how much the oracle string was helpful to you?" was maximized. $\square$ \\

To avoid bias the horacle designers should solve a lot of problems. Eventually the intention of the horacle is to make people happier whatever this means to them. But in practice maybe it's better to raise several clones in several environments to make sure that the happiness therm is not culturally biased (just for example). As the oracle can make any experiment he wishes (he already basically did it), the query itself is not that simple - yet obviously this is solvable issue. 

As for the Soracle, he is basically problem solver. He has a piece of code for every problem just like stack exchange. It might be possible to connect Soracle to a robot that will simply fix any problem which arise to the subject and being asked to fix. The question with Soracle might be economical one, if everything is done for free and without any effort, the Soracle looks like a helpful and useful tool to keep in the house but he has no "power" over the subject or interaction with him. The Soracle is just a very useful tool. 

Each of the voracles might have subjective functions as well (depending on the code generating them). If they would become subjects depending on the function $H$ installed in them they can become good or evil to humanity. It's also possible to install subjective function $H$ to a voracle that wasn't acting with any awareness. 

\section{God inside a simulation}

Now we will define several theological models that are valid in mathematical sense. In our model everything is occurring inside a simulation of physical reality indistinguishable for the observer (the subject) from the physical reality. As our basic prototyping model we assume God is a programmer inside this simulation that has unrestricted access to the tuple (some oracle, source code). This is only one theological model based on trinity of (God, Oracle, Source code). We could for example have limited access to the source code and allow direct but limited access for everyone to Horacle + Soracle (idealistic Buddhism). Or we could formulate a rule set in which programming the reality (Godhood) is divided into limited skill set that can be traded or taught just like any other skill (like in Naruto). One could approach it as a very realistic game the rules for which can be decided outside of the game, while some of the players are not noticing it's a game and believe this game is reality (and might be for them there is no distinction indeed).\\

One should also notice it's simplified model - we don't include quantum computation nor random. Our model could be extended further in the next chapters. We're currently building a way of thinking and language that has mathematical meaning, which allow to include a lot of theological entities. We then further discuss the game rooms rules in which specific theological approach can be formulated, and check compatibility with other game rooms. \\

\textit{Definition: Game Room} is a simulation of physical reality where $\forall$ subject $A$, the model of reality is indistinguishable from current reality. Unless stated otherwise by rules of the game room for some subset of subjects. $\square$ \\

By default every subject inside the game room is unaware he is inside a game room at all. The subject thinks he is inside the reality he knows and familiar with. It might be unaware groundhog day or anything we like to define, as long as the reality obeys mathematical laws and can be programmed into some virtual reality, the sentences we describe with our reality have meaning. But in order to make sure the meaning is present, we should formulate exactly how to make a game room as we like. \\

To make the game room interesting people had invented "rule breakers" something like superman for example. As it's mathematically possible to formulate a person who can fly by his wish, and there is nothing contradictory about that - we can program this rule inside some computational model of this sort i.e. to design a game room where this reality is actually real not only perceived as real. People already have no problem to imagine a game where you're a hero and you can fly but other game players can't, the problem of most people is to imagine that the current reality is basically another game rule but currently without any real known and very well documented rule breakers. Anyway as we can design such virtual games, the theological discussion about God at least some parts of this discussion can be formulated as different reality inside a game room. Using such approach we can very simply define God. \\

\textit{Definition: God} is a subject inside game room that has access to some oracle (usually preferably to him Soracle), source code of the reality which executes him, and powers to design reality as he wishes like any universal construction and computation creature inside cellular automaton. $\square$

So why do we need the oracle in the first place? Can't we just manage with God and source code alone? He only needs to simulate the reality slower than he simulates himself and to create reality preferable to him? The point is that for God there is no distinction between simulation of reality and actual reality - as he is a creature inside some sort of cellular automaton he would like to have access to the oracle to know the printed result of any infinite computation. Obviously inside this computation he had to work hard and check every possible finite string. So if you are willing to accept supertasks possibility you would be able to see that single supertask can generate such oracle, and that this oracle would be very useful to humanity. This is for example a possible physical implication from this theory. 

To make it more clear we need to define the creator-creation relationship inside our theology, as it seems the creator and the creation are basically placed on the same level just like any UTM which simulates two UTMs but it doesn't matter if time is not a factor, and although in CA there is well defined global time, the UTM can generate copies which will work twice slower than the creator and thus the creator might have more computational resources. In order to do that the creator needs to limit his creation 






\end{document}