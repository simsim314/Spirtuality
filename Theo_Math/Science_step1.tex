\documentclass{article}
\usepackage{hyperref}
\usepackage{blindtext}
\usepackage{ulem}
\usepackage[T1]{fontenc}
\usepackage[utf8]{inputenc}
\usepackage{amsmath}
\usepackage{graphicx}
\usepackage{hyperref}
\usepackage{amsfonts}
\usepackage{amsmath}
\usepackage{amssymb}
\usepackage{ mathrsfs }

\title{\Huge Identifying good science}
\author{Michael Simkin}
\begin{document}
	\maketitle
	\section{Introduction}

Karl Popper was one of the milestones in philosophy of science. We will show a mathematical proof that Karl's Popper falsifiability criteria might be necessary but not enough to make meaningful scientific statements about reality. We will then discuss alternative ways to generate a quality measurement for different scientific disciplines, and show generalized approach to Karl's Popper idea, when the falsifiability criteria is a useful yet random line to show case the scientificity measure of arbitrary statement about empirical reality. 

During the article I will assume deterministic reality inside cellular automaton and the science is done by this creature to discover its reality. Thus elegantly avoiding any more complex models like quantum wave functions etc. 

	\section{Falsifying the falsifiability criteria}
	
	\textit{Definition: Phenomenon} $A$ is a subset of reality states $\{S_{i}\}$ and a function $f_{A}$ such that for a given unseen before reality state $S'$:
	
	\begin{align*}	
	f_{A}(S') \to \{0,1\} \ \mid f(S') = 1 \iff S' \in \{S_{i}\} \ \ \ \ 	\square
	\end{align*}

	
	One should notice phenomena exist only for the subjects inside the reality, as they need to recognize patterns in meaningful and predictive way. \\
	
	\textit{Definition: Good science} is a general framework of practices which can be formulated as a list of "do" and "don't do", which when applied will eventually answer correctly to any empirical question i.e. when asked to find cause for any phenomenon, and will converge to the solution asymptotically $\square$\\

	\textit{Definition: Very good science} is a good science that will also converge to the exact solution if it can be formulated in some mathematical language. $\square$\\
	
	\textbf{Theorem 1:}  Falsifiability is not enough to make good science. \\\\
	\textbf{Prof:} Lets assume that falsifiability is enough to generate scientifically good assessments about reality. We will provide a way to generate infinite array of hypotheses $\mathcal{H}_{i}$, such that the amount of time to disprove them all will be infinite but we will not get any closer to the solution, yet every step will be concerning falsifiability (i.e. should be scientific). If we can generate an infinite array $\mathcal{H}_{i}$ of hypotheses such that every $\mathcal{H}_{i}$ will be possible to disprove in finite time, falsifiability is not enough to make good science. To generate an infinite list of false hypothesis for any empirical phenomenon $E$ I will build hypothesis $\mathcal{H}_{i}$ = \{Turn around yourself $2^{i}$ times and cause $E\}$. Notice that all the turning around yourself for hypothesis $\mathcal{H}_{i}$ is not enough turns to disprove hypothesis $\mathcal{H}_{i+1}$. $\blacksquare$\\

Although the theorem is mathematical, the construction is simplistic and formal. You can in general generate a correlation measure between turning around yourself and event $E$, and if there is no correlation we can falsify every "nice enough" functions between $\mathcal{H}_{i}$ and $E$. Now I will show stronger claim. Even if you can disclaim claims using correlation and identities of arbitrary entities (for example you can claim there is no correlation between any hypothetical human action and rain and show this claim is disprovable yet not disproved). We will construct a more sophisticated set of $\mathcal{H}_{i}$ that will basically do the same thing.\\ 

\textit{Definition: Correlation axiom} Is the assumption all causal functions between two phenomenon in nature should be revealed through linear correlation analysis. And all the functions are approximated well by Taylor series $\square$\\

\textbf{Theorem 2:}  Falsifiability is enough to make good science when we know that only phenomenon $A$ caused phenomenon $B$\\\\
\textbf{Prof:} 
As we know the cause of $B$ is $A$ and the function can be approximated with Taylor series, we can falsifiably estimate the Nth derivative of the Taylor series. As we  $\blacksquare$\\

Notice that even in the case we know that $A$ caused $B$ we can't provide any good algorithm to make very good science.\\ 

\textbf{Theorem 3:}  Given correlation axiom, falsifiability is not enough to make very good science \\\\
\textbf{Prof:} 
Lest assume phenomenon $C$ is caused by two orthogonal phenomena $A$ and $B$. As we need to generate infinite amount of hypotheses to find the function that computes $C$ from $A$ and $B$, the falsifiability gives no guarantee that we will search all Taylor coefficients of $B$. We can falsifiable calculate all Taylor coefficients of $A$ alone, computing all . Every hypothesis about correlation between yet we don't know 

exhaustive search


- not a good science epicycles
Theorem 3 Falsifibility is not enough even for good hypothesis generator. 

Theorem 4 Falsifibility is enough to describe objects. 

\end{document}

Attacking mountain optimization problem